
\documentclass{res} % This file uses the resume document class (res.cls)
\usepackage{helvetica}
\usepackage{newcent}    
\usepackage{graphicx}
\usepackage[usenames,dvipsnames]{color}
\definecolor{grey}{RGB}{211,211,211}


\setlength{\textheight}{9.5in} 
\setlength{\textwidth}{6.5in} 

\begin{document} 

\name{ KUNALKUMAR B CHAVDA\\[20pt]}    
				             

\address{\bf  ADDRESS\\1136, Himmat lal ni chali\\Opp. Bnak of Baroda\\Po. Bajwa\\Vadodara\\Gujarat-391310\\}
\address{\bf CONTACT \\ \textbf{Mo.} : +91 7874900739 \\  \textbf{E-m@il}: kunalchavda96@gmail.com}
\begin{figure}
{\includegraphics[width=3cm]{DSCN4570.JPG}}
\end{figure}

\begin{resume}

\section{
\colorbox{grey}{OBJECTIVE}
   }   
\ To work with well established technical organization for enhancement of my skills and betterment of future
 

\section{
\colorbox{grey}{EDUCATION}  
}
\begin{itemize}
\item Pursuing 8th Semester Bachelor of Engineering ( Civil Engineering )
\end{itemize}    

\begin{table}[ht] 
 \centering% used for centering table
\begin{tabular}{||c|c|c|c|c||} % centered columns (5 columns)
\hline
\hline\\ [.5ex] %inserts double horizontal lines
EXAMINATION & UNIVERSITY & INSTITUTE & YEAR & RESULT/CGPA \\ [.5ex] % inserts table
\hline
\hline\\ [.5ex]
BE SEM-7 & GTU & Vadodara Institute of Engg. & 2018 & 7.08 \\ [.5ex] % inserting body of the table
HSC & GSHSEB & Fertilizer Nagar Schhol & 2013 & 55.1{\%} \\ [.5ex] 
SSC & GSEB & Mahireva Aadarsh Vidyalaya & 2011 & 81.4{\%} \\ [1ex]
\hline %inserts single line 
\end{tabular}
\label{table:lin} % is used to refer this table in the text
\end{table}




\section{
\colorbox{grey}{PROJECTS}
}

\begin{tabular}{ l l}
\ Name &  {:} \textbf{Restoration and replenishment of tributaries of Vishwamitri river}\\ [0.5ex]
\ During & {:}  \textbf{7th and 8th semester} \\ [0.5ex]
\ Applocation  &  {:} \bf RIVER RESTORATION  \\ [0.5ex]
\ Summary & {:} \\
\end{tabular}

%-------------------------------------------------------
\             The filter is used to restore and rehabilitate tributaries of any river by filtering waste water coming from upstream side of a tributary and discharging treated water to the river. Filter is designed that it can operate itself or can be operated manually under the supervision of single operator. The filter also designed keeping in mind flooding conditions and its relative measures are manipulated by micro-controller. In high flood condition, filter will get lifted so that it does not become an obstruction to the flood flow and surrounded area will not be affected. This type of portable filter can be for any tributary of river to reduce overall polution load of river.  \par
%-------------------------------------------------------

\section{
\colorbox{grey}{TRAINING/INTERNSHIP}
}
\begin{itemize}
\item Attended workshop on “Use of Microsoft Excel in Civil Engineering” organized by EZ Professional Training Institute
\end{itemize}   






\end{resume}
\end{document}
